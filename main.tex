\documentclass[a4paper, 12pt]{article}
\usepackage[utf8]{inputenc}
\usepackage[english,russian]{babel}
\usepackage{morewrites}
\usepackage{amsmath}
\usepackage{graphicx}
\usepackage[pdf]{graphviz}
\usepackage{xpatch}
\usepackage[left=1cm, right=1cm, top=2cm, bottom=2cm, bindingoffset=0cm]{geometry}
\makeatletter
\newcommand*{\addFileDependency}[1]{% argument=file name and extension
  \typeout{(#1)}
  \@addtofilelist{#1}
  \IfFileExists{#1}{}{\typeout{No file #1.}}
}
\makeatother
\xpretocmd{\digraph}{\addFileDependency{#2.dot}}{}{}

\title{Регулярные языки и конечные автоматы}
\author{Антон Володин, гр. А-05-19}
\date{Март 2022}

\begin{document}

\maketitle

\section{Задание №1. Построить конечный автомат, распознающий язык}
Ответом на данное задание является конечный автомат, распознающий описанный язык. Автомат должен быть детерминированным.
\begin{enumerate}
  \item $L = \{w \in \{a, b, c\}^*| \left| w \right|_c = 1 \}$ \\
  
    \digraph[scale=0.7]{i1p1graph}{
        rankdir=LR;
        node [shape = none] start [label= < >];
        node [shape = doublecircle] q2;
        node [shape = circle];
        start -> q1;
        q1 -> q2 [label = "c"];
        q1 -> q1 [label = "a, b"];
        q2 -> q2 [label = "a, b"];
    } \\
  
  \item $L = \{w \in \{a, b\}^*| \left| w \right|_a \leq 2, \left| w \right|_b \geq 2 \}$ \\
  
    \digraph[scale=0.8]{i1p2graph}{
        rankdir=LR;
        node [shape = none] start [label= < >];
        node [shape = doublecircle] q22, q02, q12;
        node [shape = circle];
        start -> q00;
        q00 -> q10 [label = "a"];
        q00 -> q01 [label = "b"];
        q10 -> q20 [label = "a"];
        q10 -> q11 [label = "b"];
        q01 -> q11 [label = "a"];
        q01 -> q02 [label = "b"];
        q20 -> q21 [label = "b"];
        q11 -> q21 [label = "a"];
        q11 -> q12 [label = "b"];
        q02 -> q12 [label = "a"];
        q02 -> q02 [label = "b"];
        q21 -> q22 [label = "b"];
        q12 -> q22 [label = "a"];
        q12 -> q12 [label = "b"];
        q22 -> q22 [label = "b"];
    } \\
  
  \item $L = \{w \in \{a, b\}^*| \left| w \right|_a \neq \left| w \right|_b \}$\\
    Язык не является регулярным, поэтому невозможно построить ДКА.
    
    
  \item $L = \{w \in \{a, b\}^* | \ ww = w w w \}$ \\
  Подходит только пустой язык. \\
  
    \digraph[scale=0.7]{i1p4graph}{
        rankdir=LR;
        node [shape = none] start [label= < >];
        node [shape = doublecircle] q1;
        node [shape = circle];
        start -> q1;
    } \\
  
\end{enumerate}

\section {Задание №2. Построить конечный автомат, используя прямое произведение}
Ответом на данное задание является конечный автомат, распознающий описанный язык. Требуется, чтобы он был построен при помощи прямого произведения ДКА и его свойств.

\begin{enumerate}
  \item $L_1 = \{w \in \{a, b\}^*| \left| w \right|_a \geq 2 \land     \left| w \right|_b \geq 2\}$ \\
    ДКА, распознающий язык $L_{11} = \{w \in \{a, b\}^*| \left| w \right|_a \geq 2 \}$ \\
  
    \digraph[scale=0.7]{i2p1s1graph}{
        rankdir=LR;
        node [shape = none] start [label= < >];
        node [shape = doublecircle] q3;
        node [shape = circle];
        start -> q1;
        q1 -> q2 [label = "a"];
        q1 -> q1 [label = "b"]
        q2 -> q3 [label = "a"];
        q2 -> q2 [label = "b"];
        q3 -> q3 [label = "a,b"]
    } \\
  
    ДКА, распознающий язык $L_{12} = \{w \in \{a, b\}^*| \left| w \right|_b \geq 2\}$ \\
  
    \digraph[scale=0.7]{i2p1s2graph}{
        rankdir=LR;
        node [shape = none] start [label= < >];
        node [shape = doublecircle] q6;
        node [shape = circle];
        start -> q4;
        q4 -> q5 [label = "b"];
        q4 -> q4 [label = "a"]
        q5 -> q6 [label = "b"];
        q5 -> q5 [label = "a"];
        q6 -> q6 [label = "a, b"]
    } \\
  
    $L_1 = L_{11} \times L_{12}$ \\
    $\Sigma=\{a,b\}$ \\
    $Q = \{q1q4, q1q5, q1q6, q2q4, q2q5, q2q6, q3q4, q3q5, q3q6\}$ \\
    $S = \{q1q4\}$ \\
    $T = \{q3q6\}$ \\
    \begin{tabular} {l l l}
        $\delta (q1q4, a) = q2q4 $ & $\delta (q1q4, b) = q1q5$ & $\delta (q1q5, a) = q2q5$ \\ 
        $\delta (q1q5, b) = q1q6$ & $\delta (q1q6, a) = q2q6$ & $\delta (q1q6, b) = q1q6 $ \\ $\delta (q2q4, a) = q3q4$ & $\delta (q2q4, b) = q2q5$ & $\delta (q2q5, a) = q3q5$ \\ 
        $\delta (q2q5, b) = q2q6$ & $\delta (q2q6, a) = q3q6$ & $\delta (q2q6, b) = q2q6$ \\ $\delta (q3q4, a) = q3q4$ & $\delta (q3q4, b)= q3q5$ & $\delta (q3q5, a) = q3q5 $\\ 
        $\delta (q3q5, b) = q3q6$  & $\delta (q3q6, a) = q3q6$ & $\delta (q3q6, b) = q3q6 $
    \end{tabular}\\
  
    \digraph[scale=0.7]{i2p1graph}{
        rankdir=LR;
        node [shape = none] start [label= < >];
        node [shape = doublecircle] q3q6;
        node [shape = circle];
        start -> q1q4;
        q1q4 -> q2q4 [label = "a"];
        q1q4 -> q1q5 [label = "b"];
        q1q5 -> q2q5 [label = "a"];
        q1q5 -> q1q6 [label = "b"];
        q1q6 -> q2q6 [label = "a"];
        q1q6 -> q1q6 [label = "b"];
        q2q4 -> q3q4 [label = "a"];
        q2q4 -> q2q5 [label = "b"];
        q2q5 -> q3q5 [label = "a"];
        q2q5 -> q2q6 [label = "b"];
        q2q6 -> q3q6 [label = "a"];
        q2q6 -> q2q6 [label = "b"];
        q3q4 -> q3q4 [label = "a"];
        q3q4 -> q3q5 [label = "b"];
        q3q5 -> q3q5 [label = "a"];
        q3q5 -> q3q6 [label = "b"];
        q3q6 -> q3q6 [label = "a, b"];
  }
  
  
  
  
  
  \item $L_2 = \{w \in \{a, b\}^*| \left| w \right| \geq 3 \land \left| w \right| \text{нечётное} \}$ \\
  
      ДКА, распознающий язык $L_{21} = \{w \in \{a, b\}^*| \left| w \right| \geq 3 \}$ \\
  
    \digraph[scale=0.7]{i2p2s1graph}{
        rankdir=LR;
        node [shape = none] start [label= < >];
        node [shape = doublecircle] q4;
        node [shape = circle];
        start -> q1;
        q1 -> q2 [label = "a,b"];
        q2 -> q3 [label = "a,b"];
        q3 -> q4 [label = "a,b"];
        q4 -> q4 [label = "a,b"]
    } \\
  
      ДКА, распознающий язык $L_{22} = \{w \in \{a, b\}^*| \left| w \right| \text{нечётное}\}$ \\
  
    \digraph[scale=0.7]{i2p2s2graph}{
        rankdir=LR;
        node [shape = none] start [label= < >];
        node [shape = doublecircle] q6;
        node [shape = circle];
        start -> q5;
        q5 -> q6 [label = "a,b"];
        q6 -> q5 [label = "a,b"];
    }\\
  
    $L_2 = L_{21} \times L_{22}$ \\
    $\Sigma=\{a,b\}$ \\
    $Q = \{q1q5, q1q6, q2q5, q2q6, q3q5, q3q6, q4q5, q4q6\}$ \\
    $S = \{q1q5\}$ \\
    $T = \{q4q6$\} \\
    \begin{tabular} {l l l}
        $\delta (q1q5, a) =  q2q6 $& $\delta (q1q5, b) = q2q6$ & $\delta (q1q6, a) = q2q5$ \\ 
        $\delta (q1q6, b) = q2q5$ & $\delta (q2q5, a) = q3q6$ & $\delta (q2q5, b) = q3q6$ \\ $\delta (q3q5, a) =  q4q6$ & $\delta (q3q5, b) = q4q6$ & $\delta (q3q6, a) = q4q5$ \\ 
        $\delta (q3q6, b) =  q4q5$ & $\delta (q4q5, a) =  q4q6$ & $\delta (q4q6, b) =  q4q6$\\ 
    \end{tabular} \\
  
    \digraph[scale=0.7]{i2p2graph}{
        rankdir=LR;
        node [shape = none] start [label= < >];
        node [shape = doublecircle] q4q6;
        node [shape = circle];
        start -> q1q5;
        q1q5 -> q2q6 [label = "a,b"];
        q1q6 -> q2q5 [label = "a,b"];
        q2q5 -> q3q6 [label = "a,b"];
        q2q6 -> q3q5 [label = "a,b"];
        q3q5 -> q4q6 [label = "a,b"];
        q3q6 -> q4q5 [label = "a,b"];
        q4q5 -> q4q6 [label = "a,b"];
        q4q6 -> q4q5 [label = "a,b"];
    } \\
  
    Вершины q1q6, q2q5, q3q6 можно удалить, поскольку они недостижимы
  
      
      \item  $L_3 = \{w \in \{a, b\}^*| \left| w \right|_a \text{чётно} \land \left| w \right|_b \text{кратно трём} \}$ \\
      
      ДКА, распознающий язык $L_{31} = \{w \in \{a, b\}^*| \left| w \right|_a \text{чётно} \}$ \\
  
    \digraph[scale=0.7]{i2p3s1graph}{
        rankdir=LR;
        node [shape = none] start [label= < >];
        node [shape = doublecircle] q1;
        node [shape = circle];
        start -> q1;
        q1 -> q2 [label = "a"];
        q1 -> q1 [label = "b"];
        q2 -> q1 [label = "a"];
        q2 -> q2 [label = "b"];
    } \\
  
      ДКА, распознающий язык $L_{32} = \{w \in \{a, b\}^*| \left| w \right|_b \text{кратно трём}\}$ \\
  
    \digraph[scale=0.7]{i2p3s2graph}{
        rankdir=LR;
        node [shape = none] start [label= < >];
        node [shape = doublecircle] q3;
        node [shape = circle];
        start -> q3;
        q3 -> q3 [label = "a"];
        q3 -> q4 [label = "b"];
        q4 -> q4 [label = "a"];
        q4 -> q5 [label = "b"];
        q5 -> q5 [label = "a"];
        q5 -> q3 [label = "b"];
    }\\
  
      $L_3 = L_{31} \times L_{32}$ \\
      $\Sigma=\{a,b\}$ \\
      $Q = \{q1q3, q1q4, q1q5, q2q3, q2q4, q2q5\}$ \\
      $S = \{q1q3\}$ \\
      $T = \{q1q3$\} \\
  
      \begin{tabular} {l l l}
          $\delta (q1q3, a) = q2q3$ & $\delta (q1q3, b) =  q1q4$ & $\delta (q1q4, a) = q2q4$\\ 
          $\delta (q1q4, b) = q1q5$ & $\delta (q1q5, a) = q2q5$ & $\delta (q1q5, b) =  q1q3$\\ $\delta (q2q3, a) =  q1q3$ & $\delta (q2q3, b) = q2q4$ & $\delta (q2q4, a) = q1q4$\\ 
          $\delta (q2q4, b) = q2q5$ & $\delta (q2q5, a) = q1q5$ & $\delta (q2q5, b) =  q2q3$\\ 
      \end{tabular}\\
  
    \digraph[scale=0.6]{i2p3graph}{
        rankdir=LR;
        node [shape = none] start [label= < >];
        node [shape = doublecircle] q1q3;
        node [shape = circle];
        start -> q1q3;
        q1q3 -> q2q3 [label = "a"];
        q1q3 -> q1q4 [label = "b"];
        q1q4 -> q2q4 [label = "a"];
        q1q4 -> q1q5 [label = "b"];
        q1q5 -> q2q5 [label = "a"];
        q1q5 -> q1q3 [label = "b"];
        q2q3 -> q1q3 [label = "a"];
        q2q3 -> q2q4 [label = "b"];
        q2q4 -> q1q4 [label = "a"];
        q2q4 -> q2q5 [label = "b"];
        q2q5 -> q1q5 [label = "a"];
        q2q5 -> q2q3 [label = "b"];
  } \\
  
  
  
  
  \item $L_4 = \overline{L_3}$ \\
  
      $\Sigma=\{a,b\}$ \\
      $Q = \{q1q3, q1q4, q1q5, q2q3, q2q4, q2q5\}$ \\
      $S = \{q1q3\}$ \\
      $T = \{q1q4, q1q5, q2q3, q2q4, q2q5$\} \\
 
    \digraph[scale=0.6]{i2p4graph}{
        rankdir=LR;
        node [shape = none] start [label= < >];
        node [shape = doublecircle] q1q4, q1q5, q2q3, q2q4, q2q5;
        node [shape = circle];
        start -> q1q3;
        q1q3 -> q2q3 [label = "a"];
        q1q3 -> q1q4 [label = "b"];
        q1q4 -> q2q4 [label = "a"];
        q1q4 -> q1q5 [label = "b"];
        q1q5 -> q2q5 [label = "a"];
        q1q5 -> q1q3 [label = "b"];
        q2q3 -> q1q3 [label = "a"];
        q2q3 -> q2q4 [label = "b"];
        q2q4 -> q1q4 [label = "a"];
        q2q4 -> q2q5 [label = "b"];
        q2q5 -> q1q5 [label = "a"];
        q2q5 -> q2q3 [label = "b"];
    } \\
  
  \item $L_5 = L_2 \setminus L_3$ \\
    $L_5 = L_2 \setminus L_3 = L_2 \times L_4$ \\
    Перенумеруем вершины в графе $L_2$ и удалим лишние\\
    \digraph[scale=0.7]{i2p5s1graph}{
            rankdir=LR;
            node [shape = none] start [label= < >];
            node [shape = doublecircle] q4;
            node [shape = circle];
            start -> q1;
            q1 -> q2 [label = "a,b"];
            q2 -> q3 [label = "a,b"];
            q3 -> q4 [label = "a,b"];
            q5 -> q4 [label = "a,b"];
            q4 -> q5 [label = "a,b"];
    } \\
    Перенумеруем вершины в графе $L_4$ \\
  
    \digraph[scale=0.6]{i2p5s2graph}{
        rankdir=LR;
        node [shape = none] start [label= < >];
        node [shape = doublecircle] q8, q10, q7, q9, q11;
        node [shape = circle];
        start -> q6;
        q6 -> q7 [label = "a"];
        q6 -> q9 [label = "b"];
        q9 -> q8 [label = "a"];
        q9 -> q10 [label = "b"];
        q10 -> q11 [label = "a"];
        q10 -> q6 [label = "b"];
        q7 -> q6 [label = "a"];
        q7 -> q8 [label = "b"];
        q8 -> q9 [label = "a"];
        q8 -> q11 [label = "b"];
        q11 -> q10 [label = "a"];
        q11 -> q7 [label = "b"];
    } \\
  
  
    $Q = \{q1q6, q1q7, q1q8, q1q9, q1q10, q1q11,
             q2q6, q2q7, q2q8, q2q9, q2q10, q2q11,\\
             q3q6, q3q7, q3q8, q3q9, q3q10, q3q11,
             q3q6, q4q7, q4q8, q4q9, q4q10, q4q11,
             q5q6, \\ q5q7, q5q8, q5q9, q5q10, q5q11\}$ \\
    $S = \{q1q6\}$ \\
    $T = \{q4q7, q4q8, q4q9, q4q10, q4q11$\} \\
    \begin{tabular} {l l l}
        $\delta (q1q6, a) = q2q7 $ & $\delta (q1q6, b) =  q2q9$ & $\delta (q1q7, a) = q2q6$\\ 
        $\delta (q1q7, b) =q2q8 $ & $\delta (q1q8, a) = q2q9$ & $\delta (q1q8, b) = q2q11 $\\ $\delta (q1q9, a) =q2q8  $ & $\delta (q1q9, b) = q2q10$ & $\delta (q1q10, a) = q2q11$\\ 
        $\delta (q1q10, b) = q2q6$ & $\delta (q1q11, a) = q2q10$ & $\delta (q1q11, b) = q2q7 $\\ 
        $\delta (q2q6, a) = q3q7 $ & $\delta (q2q6, b) = q3q9  $ & $\delta (q2q7, a) = q3q6 $\\ 
        $\delta (q2q7, b) = q3q8 $ & $\delta (q2q8, a) = q3q9 $ & $\delta (q2q8, b) =  q3q11 $\\ $\delta (q2q9, a) =  q3q8 $ & $\delta (q2q9, b) = q3q10 $ & $\delta (q2q10, a) = q3q11 $\\ 
        $\delta (q2q10, b) = q3q6 $ & $\delta (q2q11, a) = q3q10 $ & $\delta (q2q11, b) = q3q7 $\\ 
        $\delta (q3q6, a) = q4q7 $ & $\delta (q3q6, b) = q4q9  $ & $\delta (q3q7, a) = q4q6 $\\ 
        $\delta (q3q7, b) = q4q8 $ & $\delta (q3q8, a) = q4q9 $ & $\delta (q3q8, b) =  q4q11 $\\ $\delta (q3q9, a) =  q4q8 $ & $\delta (q3q9, b) = q4q10 $ & $\delta (q3q10, a) = q4q11 $\\ 
        $\delta (q3q10, b) = q4q6 $ & $\delta (q3q11, a) = q4q10 $ & $\delta (q3q11, b) = q4q7 $\\ 
        $\delta (q4q6, a) = q5q7 $ & $\delta (q4q6, b) = q5q9  $ & $\delta (q4q7, a) = q5q6 $\\ 
        $\delta (q4q7, b) = q5q8 $ & $\delta (q4q8, a) = q5q9 $ & $\delta (q4q8, b) =  q5q11 $\\ $\delta (q4q9, a) =  q5q8 $ & $\delta (q4q9, b) = q5q10 $ & $\delta (q4q10, a) = q5q11 $\\ 
        $\delta (q4q10, b) = q5q6 $ & $\delta (q4q11, a) = q5q10 $ & $\delta (q4q11, b) = q5q7 $\\ 
        $\delta (q5q6, a) = q4q7 $ & $\delta (q5q6, b) = q4q9  $ & $\delta (q5q7, a) = q4q6 $\\ 
        $\delta (q5q7, b) = q4q8 $ & $\delta (q5q8, a) = q4q9 $ & $\delta (q5q8, b) =  q4q11 $\\ $\delta (q5q9, a) =  q4q8 $ & $\delta (q5q9, b) = q4q10 $ & $\delta (q5q10, a) = q4q11 $\\ 
        $\delta (q5q10, b) = q4q6 $ & $\delta (q5q11, a) = q4q10 $ & $\delta (q5q11, b) = q4q7 $\\ 
    \end{tabular}\\

    \digraph[scale=0.33]{i2p5graph}{
        rankdir=LR;
        node [shape = none] start [label= < >];
        node [shape = doublecircle] q4q7, q4q8, q4q9, q4q10, q4q11;
        node [shape = circle];
        start -> q1q6;
         q1q6 -> q2q7[label = "a"];
         q1q6 ->  q2q9 [label = "b"];      
         q1q7 -> q2q6 [label = "a"];
         q1q7 ->q2q8 [label = "b"];
         q1q8 -> q2q9 [label = "a"];   
         q1q8 -> q2q11 [label = "b"];      
         q1q9 ->q2q8 [label = "a"];        
         q1q9 -> q2q10 [label = "b"];     
         q1q10 -> q2q11 [label = "a"];  
         q1q10 -> q2q6 [label = "b"];      
         q1q11 -> q2q10 [label = "a"];      
         q1q11 -> q2q7 [label = "b"];   
         q2q6 -> q3q7 [label = "a"];      
         q2q6 -> q3q9 [label = "b"];       
         q2q7 -> q3q6 [label = "a"];   
         q2q7 -> q3q8 [label = "b"];       
         q2q8 -> q3q9 [label = "a"];       
         q2q8 ->  q3q11 [label = "b"];     
         q2q9 ->  q3q8 [label = "a"];      
         q2q9 -> q3q10 [label = "b"];     
         q2q10 -> q3q11 [label = "a"];   
         q2q10 -> q3q6 [label = "b"];      
         q2q11 -> q3q10 [label = "a"];       
         q2q11 -> q3q7 [label = "b"];    
         q3q6 -> q4q7 [label = "a"];       
         q3q6 -> q4q9 [label = "b"];        
         q3q7 -> q4q6 [label = "a"];   
         q3q7 -> q4q8 [label = "b"];      
         q3q8 -> q4q9 [label = "a"];       
         q3q8 ->  q4q11 [label = "b"];      
         q3q9 ->  q4q8 [label = "a"];       
         q3q9 -> q4q10 [label = "b"];      
         q3q10 -> q4q11 [label = "a"];   
         q3q10 -> q4q6 [label = "b"];      
         q3q11 -> q4q10 [label = "a"];       
         q3q11 -> q4q7 [label = "b"];   
         q4q6 -> q5q7 [label = "a"];      
         q4q6 -> q5q9 [label = "b"];      
         q4q7 -> q5q6 [label = "a"];   
         q4q7 -> q5q8 [label = "b"];      
         q4q8 -> q5q9 [label = "a"];      
         q4q8 ->  q5q11 [label = "b"];      
         q4q9 ->  q5q8 [label = "a"];       
         q4q9 -> q5q10 [label = "b"];       
         q4q10 -> q5q11 [label = "a"];   
         q4q10 -> q5q6 [label = "b"];      
         q4q11 -> q5q10 [label = "a"];      
         q4q11 -> q5q7 [label = "b"];    
         q5q6 -> q4q7 [label = "a"];      
         q5q6 -> q4q9 [label = "b"];        
         q5q7 -> q4q6 [label = "a"];   
         q5q7 -> q4q8 [label = "b"];       
         q5q8 -> q4q9 [label = "a"];       
         q5q8 ->  q4q11 [label = "b"];      
         q5q9 ->  q4q8 [label = "a"];       
         q5q9 -> q4q10 [label = "b"];      
         q5q10 -> q4q11 [label = "a"];   
         q5q10 -> q4q6 [label = "b"];      
         q5q11 -> q4q10 [label = "a"];       
         q5q11 -> q4q7 [label = "b"];   
    }
\end{enumerate}

\section {Задание №3. Построить минимальный ДКА по регулярному выражению}
Ответом на данное задание является минимальный ДКА, который допускает тот же язык, что описывается регулярным выражением.
\begin{enumerate}
    \item $(ab+aba)^*a$
    
        \digraph[scale=0.7]{i3p1s1graph}{
            rankdir=LR;
            node [shape = none] start [label= < >];
            node [shape = doublecircle] q2;
            node [shape = circle];
            start -> q1;
            q1 -> q2 [label= "a"];
            q1 -> q3 [label="a"];
            q3 -> q1 [label="b"];
            q3 -> q4 [label="b"];
            q4 -> q1 [label="a"];
        } \\
    
        Приведем НКА к ДКА. \\
        \begin{tabular} {|c|c|c|}
            \hline
            &a&b \\ \hline
            q1 & q2q3 & \\
            q2q3 &  & q1q4\\
            q1q4 & q1q2q3& \\
            q1q2q3 & q2q3& q1q4\\
            \hline
        \end{tabular}\\
    
        \digraph[scale=0.7]{i3p1s2graph}{
            rankdir=LR;
            node [shape = none] start [label= < >];
            node [shape = doublecircle] q2q3, q1q2q3;
            node [shape = circle];
            start -> q1;
            q1 -> q2q3 [label= "a"];
            q2q3 -> q1q4 [label="b"];
            q1q4 -> q1q2q3 [label="a"];
            q1q2q3 -> q2q3 [label="a"];
            q1q2q3 -> q1q4 [label="b"];
        } \\
    
        $k_0 : (q1, q1q4), (q2q3, q1q2q3)$ \\
        $k_1 : (q1, q1q4), (q2q3), (q1q2q3)$ \\
        $k_2 : (q1), (q1q4), (q2q3), (q1q2q3)$ \\
        Автомат минимален \\
    
    \item $a(a(ab)^*b)^*(ab)^*$
        \digraph[scale=0.7]{i3p2s1graph}{
            rankdir=LR;
            node [shape = none] start [label= < >];
            node [shape = doublecircle] q2, q3;
            node [shape = circle];
            start -> q1;
            q1 -> q2 [label= "a"];
            q2 -> q3 [label="a"];
            q3 -> q2 [label="b"];
            q2 -> q4 [label="a"];
            q4 -> q2 [label="b"];
            q4 -> q5 [label="a"];
            q5 -> q4 [label="b"];
        } \\
        
        Приведем НКА к ДКА. \\
        
        \begin{tabular} {|c|c|c|}
            \hline
            &a&b \\ \hline
            q1 & q2 & \\
            q2 & q3q4 & \\
            q3q4 & q5 & q2\\
            q5 & & q4\\
            q4 & q5 & q2\\
            \hline
        \end{tabular}\\
        
        \digraph[scale=0.7]{i3p2s2graph}{
            rankdir=LR;
            node [shape = none] start [label= < >];
            node [shape = doublecircle] q2;
            node [shape = circle];
            start -> q1;
            q1 -> q2 [label= "a"];
            q2 -> q3q4 [label="a"];
            q3q4 -> q5 [label="a"];
            q3q4 -> q2 [label="b"];
            q5 -> q4 [label="b"];
            q4 -> q5 [label="a"]
            q4 -> q2 [label="b"];
        } \\
        $k_0: (q1, q3q4, q4, q5), (q2)$ \\
        $k_1: (q1), (q3q4, q4), (q5), (q2)$ \\
        $k_2: (q1), (q3q4, q4), (q5), (q2)$ \\
        
        Перестроим автомат с учётом минимальности.\\
        \digraph[scale=0.7]{i3p2s3graph}{
            rankdir=LR;
            node [shape = none] start [label= < >];
            node [shape = doublecircle] q2;
            node [shape = circle];
            start -> q1;
            q1 -> q2 [label= "a"];
            q2 -> q3q4q4 [label="a"];
            q3q4q4 -> q5 [label="a"];
            q3q4q4 -> q2 [label="b"];
            q5 -> q3q4q4 [label="b"];
        } \\
        
    \item $(a+(a+b)(a+b)b)^*$
    \\
        \digraph[scale=0.7]{i3p3s1graph}{
            rankdir=LR;
            node [shape = none] start [label= < >];
            node [shape = doublecircle] q1;
            node [shape = circle];
            start -> q1;
            q1 -> q1 [label = "a"];
            q1 -> q2 [label = "a, b"]
            q2 -> q3 [label="a, b"];
            q3 -> q1 [label="b"];
        } \\
        
        Приведем НКА к ДКА. \\
        
        \begin{tabular} {|c|c|c|}
            \hline
            &a&b \\ \hline
            q1 & q1q2 & q2\\
            q2 & q3 & q3\\
            q1q2 & q1q2q3 & q2q3\\
            q3 & & q1\\
            q2q3 & q3 & q1q3\\
            q1q3 & q1q2 & q1q2\\
            q1q2q3 & q1q2q3 & q1q2q3\\ 
            \hline
        \end{tabular}\\
        
        \digraph[scale=0.7]{i3p3s2graph}{
            rankdir=LR;
            node [shape = none] start [label= < >];
            node [shape = doublecircle] q1, q1q2, q1q3, q1q2q3;
            node [shape = circle];
            start -> q1;
            q1 -> q1q2 [label = "a"];
            q1 -> q2 [label = "b"];
            q2 -> q3 [label = "a, b"];
            q1q2 -> q1q2q3 [label = "a"];
            q1q2 -> q2q3 [label = "b"];
            q3 -> q1 [label = "b"];
            q2q3 -> q3 [label = "a"];
            q2q3 -> q1q3 [label = "b"];
            q1q3 -> q1q2 [label = "a, b"];
            q1q2q3 -> q1q2q3 [label = "a, b"];
        } \\
        $k_0: (q2, q3, q2q3), (q1, q1q2, q1q3, q1q2q3)$ \\
        $k_1: (q2), (q2q3), (q3), (q1, q1q2), (q1q3, q1q2q3)$ \\
        $k_2: (q2), (q2q3), (q3), (q1), (q1q2), (q1q3), (q1q2q3)$ \\
        Автомат минимален \\
        
    \item $(b+c)((ab)^*c+(ba)^*)^*$
    \\
        \digraph[scale=0.7]{i3p4s1graph}{
            rankdir=LR;
            node [shape = none] start [label= < >];
            node [shape = doublecircle] q2, q5, q7;
            node [shape = circle];
            start -> q1;
            q1 -> q2 [label = "b, c"];
            q2 -> q3 [label = "a"];
            q3 -> q4 [label = "b"];
            q4 -> q5 [label = "c"];
            q4 -> q3 [label = "a"];
            q5 -> q6 [label = "b"];
            q5 -> q3 [label = "a"]
            q6 -> q7 [label = "a"];
            q7 -> q6 [label = "b"];
            q2 -> q6 [label = "b"];
            q7 -> q3 [label = "a"];
        } \\
        $k_0: (q1, q3, q4, q6), (q2, q5, q7)$ \\
        $k_1: (q1), (q3), (q4), (q6), (q2, q5, q7)$ \\
        $k_2: (q1), (q3), (q4), (q6), (q2, q5, q7)$ \\
        
        Перестроим автомат с учётом минимальности. \\
        
        \digraph[scale=0.7]{i3p4s2graph}{
            rankdir=LR;
            node [shape = none] start [label= < >];
            node [shape = doublecircle] q2q5q7;
            node [shape = circle];
            start -> q1;
            q1 -> q2q5q7 [label = "b, c"];
            q2q5q7-> q3 [label = "a"];
            q3 -> q4 [label = "b"];
            q4 -> q2q5q7 [label = "c"];
            q4 -> q3 [label = "a"];
            q2q5q7 -> q6 [label = "b"];
            q6 -> q2q5q7 [label = "a"];
        } \\
        
    
    \item $(a+b)^+(aa+bb+abab+baba)(a+b)^+$
    \\
        \digraph[scale=0.7]{i3p5s1graph}{
            rankdir=LR;
            node [shape = none] start [label= < >];
            node [shape = doublecircle] q7;
            node [shape = circle];
            start -> q1;
            q1 -> q2 [label = "a, b"];
            q2 -> q2 [label = "a, b"];
            q2 -> q3 [label = "a"];
            q3 -> q4 [label = "b"];
            q3 -> q6 [label = "a"];
            q4 -> q5 [label = "a"];
            q5 -> q6 [label = "b"];
            q6 -> q7 [label = "a,b"];
            q2 -> q8 [label = "b"];
            q8 -> q6 [label = "b"];
            q8-> q9 [label = "a"];
            q9 -> q10 [label = "b"];
            q10 -> q6 [label = "a"];
            q7 -> q7 [label="a,b"];
        } \\
        Приведём НКА к ДКА.\\
        \begin{tabular} {|c|c|c|}
            \hline
            &a&b \\ \hline
            q1 & q2 & q2\\ 
            q2 & q2q3 & q2q8\\
            q2q3 & q2q3q6 & q2q4q8\\ 
            q2q8 & q2q3q9 & q2q6q8\\ 
            q2q3q6 & q2q3q6q7 & q2q4q7\\ 
            q2q4q8 & q2q3q5q9 & q2q6q8\\ 
            q2q3q5q9 & q2q3q6 & q2q4q6q8q10\\
            q2q4q6q8q10 & q2q3q5q6q7q9 & q2q6q7q8\\
            q2q3q9 & q2q3q6 & q2q4q8q10\\ 
            q2q4q8q10 & q2q3q5q6q9 & q2q6q8\\
            q2q3q5q6q9 & q2q3q6q7 & q2q4q6q7q8q10\\
            q2q6q8 & q2q3q7q9 & q2q6q7q8\\ 
            q2q3q6q7 & q2q3q6q7 & q2q4q7q8\\
            q2q4q7 & q2q3q5q7 & q2q7q8\\
            q2q3q7q9 & q2q3q6q7 &  q2q4q7q8q10\\
            q2q6q7q8 & q2q3q7q9 & q2q6q7q8\\
            q2q4q7q8 & q2q3q5q7q9 & q2q6q7q8\\
            q2q3q5q7 & q2q3q6q7 & q2q4q6q7q8\\
            q2q7q8 & q2q3q7q9 & q2q6q7q8\\
            q2q4q7q8q10 & q2q3q5q6q7q9& q2q6q7q8\\
            q2q3q5q7q9 & q2q3q6q7& q2q4q6q7q8q10\\
            q2q4q6q7q8 & q2q3q5q7q9 & q2q6q7q8\\
            q2q3q5q6q7q9 & q2q3q6q7& q2q4q6q7q8q10\\
            q2q4q6q7q8q10 & q2q3q5q6q7q9 & q2q7q8q9\\
            q2q7q8q9 & q2q3q7q9 & q2q7q8q9\\
            \hline
        \end{tabular}\\
        \\
        \digraph[scale=0.3]{i3p5s2graph}{
            node [shape = none] start [label= < >];
            node [shape = doublecircle] q2q3q6q7, q2q4q7, q2q3q7q9, q2q6q7q8, q2q4q7q8, q2q3q5q7, q2q7q8, q2q4q7q8q10, q2q3q5q7q9, q2q4q6q7q8, q2q3q5q6q7q9, q2q4q6q7q8q10, q2q7q8q9;
            node [shape = circle];
            start -> q1;
            q1 -> q2 [label = "a, b"];
            q2 -> q2q3 [label = "a"];
            q2 -> q2q8 [label = "b"];
            q2q3 -> q2q3q6 [label = "a"];
            q2q3 -> q2q4q8 [label = "b"];
            q2q8 -> q2q3q9 [label = "a"];
            q2q8 -> q2q6q8 [label = "b"];
            q2q3q6 -> q2q3q6q7 [label = "a"];
            q2q3q6 -> q2q4q7[label = "b"];
            q2q4q8  -> q2q3q5q9 [label = "a"];
            q2q4q8  -> q2q6q8 [label = "b"];
            q2q3q5q9 -> q2q3q6 [label = "a"];
            q2q3q5q9 -> q2q4q6q8q10 [label = "b"];
            q2q4q6q8q10 -> q2q3q5q6q7q9[label = "a"];
            q2q4q6q8q10 -> q2q6q7q8[label = "b"];
            q2q3q9 -> q2q3q6[label = "a"];
            q2q3q9 -> q2q4q8q10[label = "b"];
            q2q6q8 -> q2q3q7q9[label = "a"];
            q2q6q8 -> q2q6q7q8[label = "b"];
            q2q3q6q7 -> q2q3q6q7 [label = "a"];
            q2q3q6q7 -> q2q4q7q8 [label = "b"];
            q2q4q7 -> q2q3q5q7 [label = "a"];
            q2q4q7 -> q2q7q8[label = "b"];
            q2q3q7q9 -> q2q3q6q7[label = "a"];
            q2q3q7q9 -> q2q4q7q8q10[label = "b"];
            q2q6q7q8 -> q2q3q7q9 [label = "a"];
            q2q6q7q8 -> q2q6q7q8 [label = "b"];
            q2q4q7q8  -> q2q3q5q7q9[label = "a"];
            q2q4q7q8  -> q2q6q7q8[label = "b"];
            q2q3q5q7 -> q2q3q6q7 [label = "a"];
            q2q3q5q7 -> q2q4q6q7q8 [label = "b"];
            q2q7q8  -> q2q3q7q9 [label = "a"];
            q2q7q8  -> q2q6q7q8 [label = "b"];
            q2q4q7q8q10 -> q2q3q5q6q7q9[label = "a"];
            q2q4q7q8q10 -> q2q6q7q8 [label = "b"];
            q2q3q5q7q9 -> q2q3q6q7 [label = "a"];
            q2q3q5q7q9 -> q2q4q6q7q8q10[label = "b"];
            q2q4q6q7q8 -> q2q3q5q7q9 [label = "a"];
            q2q4q6q7q8 -> q2q6q7q8 [label = "b"];
            q2q3q5q6q7q9 -> q2q3q6q7 [label = "a"];
            q2q3q5q6q7q9 -> q2q4q6q7q8q10 [label = "b"];
            q2q4q6q7q8q10 -> q2q3q5q6q7q9 [label = "a"];
            q2q4q6q7q8q10 -> q2q7q8q9 [label = "b"];
            q2q7q8q9 -> q2q3q7q9 [label = "a"];
            q2q7q8q9 -> q2q7q8q9 [label = "b"];
            q2q4q8q10 -> q2q3q5q6q9 [label = "a"];
            q2q4q8q10 -> q2q6q8 [label = "b"];
            q2q3q5q6q9 -> q2q3q6q7 [label = "a"];
            q2q3q5q6q9 -> q2q4q6q7q8q10 [label = "b"];
        } \\

        $k_0: (q1, q2, q2q3, q2q8, q2q3q6, q2q4q8, q2q3q5q9, q2q4q6q8q10, q2q3q9,  q2q4q8q10, q2q3q5q6q9, q2q6q8), \\
        (q2q3q6q7, q2q4q7, q2q3q7q9, q2q6q7q8, q2q4q7q8, q2q3q5q7, q2q7q8, q2q4q7q8q10, q2q3q5q7q9, q2q4q6q7q8, \\
        q2q3q5q6q7q9, q2q4q6q7q8q10, q2q7q8q9)$ \\
        $k_1: (q1, q2), (q2q8), (q2q3q5q9, q2q3q9, q2q3, q2q4q8,),(q2q3q6, q2q3q5q6q9, q2q6q8, q2q4q6q8q10), \\
        (q2q3q6q7, q2q4q7, q2q3q7q9, q2q6q7q8, q2q4q7q8, q2q3q5q7, q2q7q8, q2q4q7q8q10, q2q3q5q7q9, q2q4q6q7q8, \\
        q2q3q5q6q7q9, q2q4q6q7q8q10, q2q7q8q9)$\\
        $k_2: (q1), (q2), (q2q8), (q2q3q5q9, q2q3q9, q2q3, q2q4q8,),(q2q3q6, q2q3q5q6q9, q2q6q8, q2q4q6q8q10), \\
        (q2q3q6q7, q2q4q7, q2q3q7q9, q2q6q7q8, q2q4q7q8, q2q3q5q7, q2q7q8, q2q4q7q8q10, q2q3q5q7q9, q2q4q6q7q8, \\
        q2q3q5q6q7q9, q2q4q6q7q8q10, q2q7q8q9)$\\
\end{enumerate}


\section{Задание №4. Определить является ли язык регулярным или нет}
Ответом на данное задание является конечный автомат, если язык регулярный, либо доказательство нерегулярности языка при помощи леммы о разрастании.
\begin{enumerate}
    \item $L=\{(aab)^n b(aba)^m \ | \ n \geq 0, m \geq 0\}$
    \\
    Язык регулярный \\ 
    \digraph[scale=0.7]{i4p1graph}{
        rankdir=LR;
        node [shape = none] start [label= < >];
        node [shape = doublecircle] q5, q8;
        node [shape = circle];
        start -> q1;
        q1 -> q2 [label = "a"];
        q2 -> q3 [label = "a"];
        q3 -> q4 [label = "b"];
        q4 -> q2 [label = "a"];
        q4 -> q5 [label = "b"];
        q1 -> q5 [label = "b"];
        q5 -> q6 [label = "a"];
        q6 -> q7 [label = "b"];
        q7 -> q8 [label = "a"];
        q8 -> q5 [label = "a"];
    } \\
    
    
    \item $L=\{uaav \ | \ u \in \{a,b\}^*, v \in \{a,b\}^*, \left| u \right|_b \geq  \left| v \right|_a\}$\\
    $w=b^naaa^n$\\
    $x=b^i$ \\
    $y=b^j,\ i+j < n, \ j>0$\\
    $z=b^{n-i-j}aaa^n$\\
    $w=xy^kz$ \\
    При $l=0 \quad w=b^{n-j}aaa^n \notin L \Rightarrow $ язык нерегулярный\\
    
    \item $L=\{ a^m w \ | \ w \in \{a, b\}^*, 1 \leq \left| w \right|_b \leq m\}$
    \\
    $w = a^n b^n, \ |w| \geq n$\\
    $x = a^i, \ |xy| \leq n, \ i \geq 0,\ j > 0$\\
    $y = a^j$\\
    $z = a^{n-i-j}b^n$\\
    При $l=0 \quad w=a^{n-j}b^n \notin L \Rightarrow $ язык нерегулярный\\
    
    \item $L=\{a^k b^m a^n \ | \  k=n \lor m>0\}$
    \\
    $w = a^n b a^n, \ |w| \geq n$ \\
    $x = a^i, \ |xy| \leq n  => i + j \leq n,\; j > 0$\\
    $y = a^j$\\
    $z = a^{n-i-j}ba^n$\\
    При $k = 2 \quad xy^2z = a^ia^{2j}a^{n-i-j}ba^n = a^{n+j}ba^n \notin L\Rightarrow $ язык нерегулярный\\
    
    \item $L=\{ucv \ | \ u \in \{a,b\}^*, v \in \{a,b\}^*, u\neq v^R  \}$
    \\
    $w = (ab)^nc(ab)^n, \ |w| \geq n$\\
    $x = \alpha_1\alpha_2...\alpha_i, \ |xy| \leq n, \  i+j \leq n, \ j \neq 0 $ \\
    $y = \alpha_{i+1}\alpha_{i+2}...\alpha_{i+j}$ \\
    $z = \alpha_{i+j+1}\alpha_{i+j+2}...\alpha_{2n} c(ab)^n$\\
    При $k = 2 \quad  xy^2z(\alpha_1\alpha_2...\alpha_i)(\alpha_{i+1}\alpha_{i+2}...\alpha_{i+j})^2(\alpha_{i+j+1}\alpha_{i+j+2}...\alpha_{2n} c (ab)^n) \notin L \Rightarrow $ язык нерегулярный\\
    
\end{enumerate}


\section{Задание №5. Реализовать алгоритмы}
Ответом на данное задание является работающая программа на выбранном языке программирования, покрытая юнит-тестами. \\

В рамках своего выполнения программма должна генерировать текстовый документ с картинками, показывающий процесс построения автомата (к примеру Markdown с графиками на Graphiz).\\
\begin{enumerate}
    \item Построение ДКА по НКА с 
    $\lambda$-переходами.
    \item Прямое произведение языков, с возможностью построить пересечение, объединение и разность.
\end{enumerate}
\end{document}
